\documentclass[11pt]{article}

\usepackage{pstricks,pst-node,amsmath,algorithm}
\usepackage[noend]{algpseudocode}

\pagestyle{empty}
\textwidth	165mm
\textheight	248mm
\topmargin	-13mm
\oddsidemargin	-2mm
\evensidemargin	-2mm


\newcommand{\id}[1]{\mbox{\textit{#1}}}
\newcommand{\tuple}[1]{\langle #1 \rangle}

% For use in graphs in pspicture environment
% \Gnode{x}{y}{name}{label}
\newcommand{\Gnode}[4]{\Cnode(#1,#2){#3}\rput(#1,#2){#4}}
\newcommand{\Hnode}[3]{\Cnode(#1,#2){#3}\rput(#1,#2){~}}
\algnewcommand{\AND}{\,\textbf{and}\,}
\algnewcommand{\TRUE}{\,\texttt{true}\,}
\algnewcommand{\FALSE}{\,\texttt{false}\,}
\begin{document}

\begin{center}
{\sc The University of Melbourne
\\
School of Computing and Information Systems
\\
COMP90038 Algorithms and Complexity}
\bigskip \\
{\Large\bf Assignment 1, Semester 2, 2017}
\bigskip \\
{\large Released: 9 August.  Deadline: 31 August at 23:00}
\bigskip\\
Tiancong Li
\end{center}

\begin{enumerate}
\item
Consider the recurrence relation $T(n) = 2 T(n/2) + 1$, with $T(1) = 1$.
\begin{enumerate}
\item
Use the Master Theorem to find a correct $\Theta$ expression for $T(n)$.
\\
\textsc{answer}:
\par
As the Master Theorem described, for a recurrence relation
$T(n)=a\;T\!\left({\frac {n}{b}}\right)+f(n)$,
\\when $f(n)=O(n^{c})$ and $c<c_{crit}$ where $c_{crit}=log_ba$,
\\we have: $T(n)=\Theta(n^{c_{crit}})$.
\par
So, for the specific relation $T(n)=2T(n/2)+1$,
\\
we have: $c_{crit}=log_ba=1$ and $f(n)=1$.
\\
Let $c=0$,
\\then we have: $f(n)=O(n^c)=O(1)$ and $0=c<c_{crit}=1$.
\\
So, we have: $T(n)=\Theta(n^{c_{crit}})=\Theta(n)$.


\item
Use telescoping to give an exact closed-form definition of $T(n)$.
\\
\textsc{answer}:
\par
Let $n=2^k$ and $S(k)=T(2^k)$,
\\so we have: $T(2^k)=2T(2^{k-1})+1$,
\\and so: $S(k)=2S(k-1)+1$.
\\Multiply both sides by $2^{-k}$,
\\so we have: $2^{-k}S(k)=2^{-(k-1)}+2^{-k}$.
\\Let $U(k)=2^{-k}S(k)$,
\\so we have: $U(k)=U(k-1)+2^{-k}$.
\\This telesopes yielding: $U(k)=U(0)+1-2^{-k}$,
\\where $U(0)=S(0)=T(1)=1$,
\\so we have: $U(k)=2-2^{-k}$,
\\and so: $2^{-k}S(k)=2-2{-k}$,
\\and so: $T(2^k)=2^(k+1)-1$.
\\Since $k=logn$,
\\so we have: $T(n)=2^{logn}*2-1=2n-1$.


\item
Suppose somebody presents you with the following argument, to show
that $T(n) \not\in \Theta(n)$.
\begin{quote}
If we had $T(n) \in \Theta(n)$ then $T(n) \leq k \cdot n$ for some constant $k$.
But, substituting into the recurrence relation, we get
$2 \cdot k \cdot n/2 + 1 \leq k \cdot n$.
And that in turn is clearly false for all $k$.
Hence we must have $T(n) \not\in \Theta(n)$.
\end{quote}
Do you agree with this argument?
Why or why not?
\\
\textsc{answer}:
\par
Disagree.
\\If $T(n)\in\Theta(n)$,
\\Then we have: $T(n) \leq k \cdot n$,
\\and so: $T(n/2) \leq k \cdot n/2$.
\\So we have: $T(n)=2T(n/2)+1 \leq 2 \cdot k \cdot n/2 + 1$.
\\So there is no way to get $2 \cdot k \cdot n/2 + 1 \leq k \cdot n$
\\from $T(n) \leq k \cdot n$ and $T(n)\leq 2 \cdot k \cdot n/2 + 1$.
\end{enumerate}

\item
DNA nucleobases are
\texttt{A} (adenine),
\texttt{C} (cytosine),
\texttt{G} (guanine) and
\texttt{T} (thymine).
Consider the problem of counting, in a DNA string $s$,
the number of sequences (substrings)
that start with an \verb!A! and end with a \verb!C!.
For example, there are four such sequences in \verb!TACAAGCGA!.
\begin{enumerate}
\item
Using pseudo-code, give a brute-force algorithm to solve this problem.
\\
\textsc{answer}:
\par
\begin{algorithmic}
\Function{counting}{$s,n,start,end$}\Comment{count substrings in s with size n}
\State $count\gets 0$
\For{$i\gets0,n-2$}
  \If{$s[i]=start$}
    \For{$j\gets i+1,n-1$}
      \If{$s[j]=end$}
        \State $count\gets scount+1$
      \EndIf
    \EndFor
  \EndIf
\EndFor
\Return $count$
\EndFunction
\end{algorithmic}
\item
Design a method that solves the problem with a single left-to-right
scan of the string $s$ and present it using pseudo-code.
Briefly explain why the complexity is linear in $|s|$.
\\
\textsc{answer}:
\par
\begin{algorithmic}
\Function{counting}{$s,n,start,end$}
\State $count \gets 0$
\State $countStart \gets 0$
\For{$i \gets 0,n-1$}
  \If{$s[i]=start$}
    \State $countStart \gets countStart + 1$
  \ElsIf{$s[i]=end$}
    \State $count \gets count + countStart$
  \EndIf
\EndFor
\Return $count$
\EndFunction
\end{algorithmic}
By counting the number of the starting character, we can get the total count of
substrings by adding current $countStart$ to the total $count$ everytime
when an ending character is found, and this process only scans the string one time.
So, the complexity is linear in $|s|$
\end{enumerate}

\item
Given two nodes $v$ and $u$ in a connected, undirected graph
$G = \tuple{V,E}$,
their \emph{distance} is the length of the shortest path between
$v$ and $u$ (here each edge is of length 1).
The \emph{remoteness} of a node is its distance to the node furthest away,
that is, $\id{rem}(v) = \max \{\id{dist}(v,u) \mid u \in V\}$.
A \emph{hub} in $G$ is a node which has minimal
remoteness, compared to all other nodes.
That is, $v$ is a hub iff
$\id{rem}(v) = \min \{\id{rem}(u) \mid u \in V\}$.
For example, $b$ is a hub in the graph on the left (below),
whereas $b$, $d$ and $e$ all are hubs in the graph on the right.
Note that a connected graph always has at least one hub.

\begin{center}
\begin{pspicture}(0,-0.2)(6,1.3)
\psset{nodesep=1pt,linecolor=black,arrows=-}

\Gnode{0}{1}{A}{$a$}
\Gnode{1}{1}{B}{$b$}
\Gnode{2}{1}{C}{$c$}
\Gnode{0}{0}{D}{$d$}
\Gnode{1}{0}{E}{$e$}
\Gnode{2}{0}{F}{$f$}

\Gnode{4}{1}{A1}{$a$}
\Gnode{5}{1}{B1}{$b$}
\Gnode{6}{1}{C1}{$c$}
\Gnode{4}{0}{D1}{$d$}
\Gnode{5}{0}{E1}{$e$}
\Gnode{6}{0}{F1}{$f$}

\ncline{A}{B}
\ncline{A}{D}
\ncline{B}{C}
\ncline{B}{D}
\ncline{B}{E}
\ncline{B}{F}
\ncline{D}{E}

\ncline{A1}{B1}
\ncline{A1}{D1}
\ncline{B1}{C1}
\ncline{B1}{D1}
\ncline{B1}{E1}
\ncline{D1}{E1}
\ncline{E1}{F1}

\end{pspicture}
\end{center}
\begin{enumerate}
\item
Using pseudo-code,
design a function that, given a connected undirected graph $\tuple{V,E}$
and a node $v \in V$, determines all distances from $v$, that is,
for each $u \in V$, gives $\id{dist}(v,u)$.
Assume that nodes are labelled 1 to $n$, so that your function
can return an array $\id{dist}$ with $\id{dist}[i]$ giving
$i$'s distance to $v$.
The algorithm should run in time that is linear in the size of the graph.
\\
\textsc{answer}:
\par
\begin{algorithmic}
\Function{getDistances}{$\tuple{V,E}, v, n$}
\State $initArray(dist, n)$\Comment{create an all zero array}
\State $initQueue(queue)$\Comment{create an empty queue}
\State $inject(queue,v)$
\While{$queue$ is non-empty}
  \State $u \gets eject(queue)$
  \For{each edge$(u, w)$}
    \If{$dist[w]=0 \AND w \neq v$}
      \State $dist[w] \gets dist[u] + 1$
      \State $inject(queue, w)$
    \EndIf
  \EndFor
\EndWhile
\Return $dist$
\EndFunction
\end{algorithmic}
Use BFS to scan the graph starting from $v$,
never go back to any nodes already discovered.
So, after one single scan, the shortest distance could be found.

\item
Design an algorithm that, given a connected undirected graph $\tuple{V,E}$,
identifies a hub (any hub) in the graph.
More precisely, use pseudo-code to define a function \textsc{Hub}
that takes $\tuple{V,E}$ as input and returns a node which is a hub.
Aim for an $O(n^2)$ algorithm, where $n$ is the size of the graph.
\\
\textsc{answer}:
\par
\begin{algorithmic}
\Function{getHub}{$\tuple{V,E}, n$}
\State $initArray(remotes, n)$\Comment{create an all zero array}
\For{each $v$ in $V$}
  \State $dist=\Call{getDistances}{\tuple{V,E}, v, n}$
  \State $remotes[v]=\Call{maxDistance}{dist, n}$
\EndFor
\Return $\Call{minRemoteNode}{remotes, n}$
\EndFunction

\Function{maxDistance}{$dist, n$}
\State $max \gets 0$
\For{$i \gets 1,n$}
  \If{$dist[i]>max$}
    \State $max=dist[i]$
  \EndIf
\EndFor
\Return $max$
\EndFunction

\Function{minRemoteNode}{$remotes, n$}
\State $min \gets remotes[1]$
\State $node = 1$
\For{$i \gets 2,n$}
  \If{$remotes[i]<min$}
    \State $min \gets remotes[i]$
    \State $node \gets i$
  \EndIf
\EndFor
\Return $node$
\EndFunction
\end{algorithmic}
Get the HUB by its definition.
Since $getDistance$ and $maxDistance$ are both $O(n)$,
after traverse each $v$, the complexity is $O(n^2)$.
\end{enumerate}

\item
Let $\tuple{V,E}$ be a directed graph.
A node $v \in V$ is an \emph{attractor} iff $v$ is a sink, and
moreover, for every node $u \in V$ with $u \not= v$, $(u,v) \in E$.
Note that a graph can have at most one attractor.
We want an algorithm that will identify an attractor in an
input graph $\tuple{V,E}$ if there is one.
Assume that a graph is represented as an adjacency matrix $A$,
with the nodes in $V$ labelled 1 to $n$.
The function $\textsc{Attractor}(A[.,.],n)$ should return $i$ if
node $i$ is an attractor, and 0 if the graph has no attractor.
\begin{enumerate}
\item
Give an $O(n^2)$ algorithm for the problem, using pseudo-code.
\\
\textsc{answer}:
\par
\begin{algorithmic}
\Function{attractor}{$A,n$}
\State $attractor \gets 0$
\State $initArray(mark, n)$\Comment{create an all zero array}
\State $initArray(attract, n)$\Comment{create an all zero array}
\State $initQueue(queue)$\Comment{create an empty queue}
\For{$i \gets 1,n$}
  \If{$mark[i]=0$}
    \State $inject(queue, i)$
    \State $mark[i] \gets 1$
    \While{$queue$ is non-empty}
      \State $u \gets eject(queue)$
      \For{$j \gets 1, n$}
        \If{$A[u][j]=1$}
          \State $attract[u] \gets attract[u]-1$
          \State $attract[j] \gets attract[j]+1$
          \If{$mark[j] = 0$}
            \State $inject(queue,j)$
            \State $mark[j] \gets 1$
          \EndIf
        \EndIf
      \EndFor
    \EndWhile
  \EndIf
\EndFor
\For{$i \gets 1,n$}
  \If{$attract[i]=n-1$}
    \State $attractor \gets i$
  \EndIf
\EndFor
\If{$A[attractor][attractor]=1$}\Comment{attractor cannot be self connected}
  \State $attractor \gets 0$
\EndIf
\Return $attractor$
\EndFunction
\end{algorithmic}
By using an $attract$ array, for every node in the graph,
when an outgoing edge is found, $-1$ is added to the corresponding bit,
and when an incoming edge is found, $1$ is added.
So, the $attractor$ is the one which hold the value of $n-1$ at the end.
The algorithm uses BFS, and its complexity is $O(n^2)$

\item
(Harder.)
Give an $O(n)$ algorithm for the problem, using pseudo-code.
Maximum marks will (only) be given for a solution that is both
readable, intelligible, carefully explained and analysed,
and, importantly, runs in linear time.
\\
\textsc{answer}:
\par
\begin{algorithmic}
\Function{attractor}{$A,n$}
\State $attractor \gets 0$
\State $i \gets 1$
\State $j \gets 1$
\While{$i \leq n \AND j \leq n$}
  \If{$A[i][j]=1$}
    \State $i \gets i+1$
  \Else
    \State $j \gets j+1$
  \EndIf
\EndWhile
\If{$i \leq n \And \Call{isAttractor}{A,i,n}$}
  \State $attractor \gets i$
\EndIf
\Return $attractor$
\EndFunction

\Function{isAttractor}{$A,i,n$}
\For{$j \gets 1,n$}
  \If{$A[i][j]=1$}
    \Return $\FALSE$
  \EndIf
  \If{$A[j][i]=0 \AND i \neq j$}
    \Return $\FALSE$
  \EndIf
\EndFor
\Return $\TRUE$
\EndFunction
\end{algorithmic}
This algorithm uses "decrease and conquer" principle.
We change the problem of finding the attractor of graph with size of $n$
to the problem that prune $n-1$ vertices which are not attractor
and check whether the remaining one is an attractor.
\\
As shown below, if the 2nd vertice is the attractor,
the matrix should looks like this,
according to the definition of attractor.
And when we arrive the end along with the path described in the above algorithm,
every vertice except the 2nd one could be pruned,
\\since:
if $A[i][2] = 1, i \neq 2$, then $i$ could not be the attractor.
Likewise, if $A[2][i] = 0, i \neq 2$, then $i$ could not be the attractor.
To be more detailed, an attractor $i$ must satisfy that $A[i][j] = 0$,
for every node $j \in V$ and $A[j][i] = 1$, for every $j \in V$ with $i \neq j$

So, after the check for the 2nd node, we could know whether there is an attractor,
which is the 2nd vertice here, or not.

There are at most $2n$ operations for pruning, and $n$ operations for checking,
so the complexity is $O(n)$.
\begin{equation*}
\begin{pmatrix}
      a_{11} & 1 & a_{13}& \dots & a_{1n}\\
      0 & 0 & 0&\dots & 0\\
      a_{31} & 1 &a_{33}& \dots & a_{3n}\\
      \hdotsfor{4}\\
      a_{n1} & 1 &a_{n3}& \dots & a_{nn}
\end{pmatrix}
\end{equation*}
\end{enumerate}

\end{enumerate}



\begin{flushright}
Tiancong Li
\\ 24 August 2017
\end{flushright}

\end{document}
